%%%%%%%%%%%%%%%%%%%%%%%%%%%%%%%%%%%%%%%%%%%%%%%%%%%%%%%%%%%%%%%%%%%%%%%%%%%%%%%%
%%%%%%%%%%%%%%%%%%%%%%%%%%%%%%%%%%%%%%%%%%%%%%%%%%%%%%%%%%%%%%%%%%%%%%%%%%%%%%%%
%%                                                                            %%
%% thesistemplate.tex version 3.20 (2018/08/31)                               %%
%% The LaTeX template file to be used with the aaltothesis.sty (version 3.20) %%
%% style file.                                                                %%
%% This package requires pdfx.sty v. 1.5.84 (2017/05/18) or newer.            %%
%%                                                                            %%
%% This is licensed under the terms of the MIT license below.                 %%
%%                                                                            %%
%% Written by Luis R.J. Costa.                                                %%
%% Currently developed at the Learning Services of Aalto University School of %%
%% Electrical Engineering by Luis R.J. Costa since May 2017.                  %%
%%                                                                            %%
%% Copyright 2017-2018, by Luis R.J. Costa, luis.costa@aalto.fi,              %%
%% Copyright 2017-2018 Swedish translations in aaltothesis.cls by Elisabeth   %%
%% Nyberg, elisabeth.nyberg@aalto.fi and Henrik Wallén,                       %%
%% henrik.wallen@aalto.fi.                                                    %%
%% Copyright 2017-2018 Finnish documentation in the template opinnatepohja.tex%%
%% by Perttu Puska, perttu.puska@aalto.fi, and Luis R.J. Costa.               %%
%% Copyright 2018 English template thesistemplate.tex by Luis R.J. Costa.     %%
%% Copyright 2018 Swedish template kandidatarbetsbotten.tex by Henrik Wallen. %%
%%                                                                            %%
%% Permission is hereby granted, free of charge, to any person obtaining a    %%
%% copy of this software and associated documentation files (the "Software"), %%
%% to deal in the Software without restriction, including without limitation  %%
%% the rights to use, copy, modify, merge, publish, distribute, sublicense,   %%
%% and/or sell copies of the Software, and to permit persons to whom the      %%
%% Software is furnished to do so, subject to the following conditions:       %%
%% The above copyright notice and this permission notice shall be included in %%
%% all copies or substantial portions of the Software.                        %%
%% THE SOFTWARE IS PROVIDED "AS IS", WITHOUT WARRANTY OF ANY KIND, EXPRESS OR %%
%% IMPLIED, INCLUDING BUT NOT LIMITED TO THE WARRANTIES OF MERCHANTABILITY,   %%
%% FITNESS FOR A PARTICULAR PURPOSE AND NONINFRINGEMENT. IN NO EVENT SHALL    %%
%% THE AUTHORS OR COPYRIGHT HOLDERS BE LIABLE FOR ANY CLAIM, DAMAGES OR OTHER %%
%% LIABILITY, WHETHER IN AN ACTION OF CONTRACT, TORT OR OTHERWISE, ARISING    %%
%% FROM, OUT OF OR IN CONNECTION WITH THE SOFTWARE OR THE USE OR OTHER        %%
%% DEALINGS IN THE SOFTWARE.                                                  %%
%%                                                                            %%
%%                                                                            %%
%%%%%%%%%%%%%%%%%%%%%%%%%%%%%%%%%%%%%%%%%%%%%%%%%%%%%%%%%%%%%%%%%%%%%%%%%%%%%%%%
%%                                                                            %%
%%                                                                            %%
%% An example for writting your thesis using LaTeX                            %%
%% Original version and development work by Luis Costa, changes to the text   %% 
%% in the Finnish template by Perttu Puska.                                   %%
%% Support for Swedish added 15092014                                         %%
%% PDF/A-b support added on 15092017                                          %%
%% PDF/A-2 support added on 24042018                                          %%
%%                                                                            %%
%% This example consists of the files                                         %%
%%         thesistemplate.tex (version 3.20) (for text in English)            %%
%%         opinnaytepohja.tex (version 3.20) (for text in Finnish)            %%
%%         kandidatarbetsbotten.tex (version 1.00) (for text in Swedish)      %%
%%         aaltothesis.cls (versio 3.20)                                      %%
%%         kuva1.eps (graphics file)                                          %%
%%         kuva2.eps (graphics file)                                          %%
%%         kuva1.jpg (graphics file)                                          %%
%%         kuva2.jpg (graphics file)                                          %%
%%         kuva1.png (graphics file)                                          %%
%%         kuva2.png (graphics file)                                          %%
%%         kuva1.pdf (graphics file)                                          %%
%%         kuva2.pdf (graphics file)                                          %%
%%                                                                            %%
%%                                                                            %%
%% Typeset in Linux either with                                               %%
%% pdflatex: (recommended method)                                             %%
%%             $ pdflatex thesistemplate                                      %%
%%             $ pdflatex thesistemplate                                      %%
%%                                                                            %%
%%   The result is the file thesistemplate.pdf that is PDF/A compliant, if    %%
%%   you have chosen the proper \documenclass options (see comments below)    %%
%%   and your included graphics files have no problems.
%%                                                                            %%
%% Or                                                                         %%
%% latex: (this method is not recommended)                                    %%
%%             $ latex thesistemplate                                         %%
%%             $ latex thesistemplate                                         %%
%%                                                                            %%
%%   The result is the file thesistemplate.dvi, which is converted to ps      %%
%%   format as follows:                                                       %%
%%                                                                            %%
%%             $ dvips thesistemplate -o                                      %%
%%                                                                            %%
%%   and then to pdf as follows:                                              %%
%%                                                                            %%
%%             $ ps2pdf thesistemplate.ps                                     %%
%%                                                                            %%
%%   This pdf file is not PDF/A compliant. You must must make it so using,    %%
%%   e.g., Acrobat Pro or PDF-XChange.                                        %%
%%                                                                            %%
%%                                                                            %%
%% Explanatory comments in this example begin with the characters %%, and     %%
%% changes that the user can make with the character %                        %%
%%                                                                            %%
%%%%%%%%%%%%%%%%%%%%%%%%%%%%%%%%%%%%%%%%%%%%%%%%%%%%%%%%%%%%%%%%%%%%%%%%%%%%%%%%
%%%%%%%%%%%%%%%%%%%%%%%%%%%%%%%%%%%%%%%%%%%%%%%%%%%%%%%%%%%%%%%%%%%%%%%%%%%%%%%%
%%
%% WHAT is PDF/A
%%
%% PDF/A is the ISO-standardized version of the pdf. The standard's goal is to
%% ensure that he file is reproducable even after a long time. PDF/A differs
%% from pdf in that it allows only those pdf features that support long-term
%% archiving of a file. For example, PDF/A requires that all used fonts are
%% embedded in the file, whereas a normal pdf can contain only a link to the
%% fonts in the system of the reader of the file. PDF/A also requires, among
%% other things, data on colour definition and the encryption used.
%% Currently three PDF/A standards exist:
%% PDF/A-1: based on PDF 1.4, standard ISO19005-1, published in 2005.
%%          Includes all the requirements essential for long-term archiving.
%% PDF/A-2: based on PDF 1.7, standard ISO19005-2, published in 2011.
%%          In addition to the above, it supports embedding of OpenType fonts,
%%          transparency in the colour definition and digital signatures.
%% PDF/A-3: based on PDF 1.7, standard ISO19005-3, published in 2012.
%%          Differs from the above only in that it allows embedding of files in
%%          any format (e.g., xml, csv, cad, spreadsheet or wordprocessing
%%          formats) into the pdf file.
%% PDF/A-1 files are not necessarily PDF/A-2 -compatible and PDF/A-2 are not
%% necessarily PDF/A-1 -compatible.
%% All of the above PDF/A standards have two levels:
%% b: (basic) requires that the visual appearance of the document is reliably
%%    reproduceable.
%% a (accessible) in addition to the b-level requirements, specifies how
%%   accessible the pdf file is to assistive software, say, for the physically
%%   impaired.
%% For more details on PDF/A, see, e.g., https://en.wikipedia.org/wiki/PDF/A
%%
%%
%% WHICH PDF/A standard should my thesis conform to?
%%
%% Primarily to the PDF/A-1b standard. All the figures and graphs typically
%% use in thesis work do not require transparency features, a basic '2-D'
%% visualisation suffices. The font to be used are specified in this template
%% and they should not be changed. However, if you have figures where
%% transparency characteristics matter, use the PDF/A-2b standard. Do not use
%% the PDF/A-3b standard for your thesis.
%%
%%
%% WHAT graphics format can I use to produce my PDF/A compliant file?
%%
%% When using pdflatex to compile your work, use jpg, png or pdf files. You may
%% have PDF/A compliance problems with figures in pdf format. Do not use PDF/A
%% compliant graphics files.
%% If you decide to use latex to compile your work, the only acceptable file
%% format for your figure is eps. DO NOT use the ps format for your figures.

%% USE one of these:
%% * the first when using pdflatex, which directly typesets your document in the
%%   chosen pdf/a format and you want to publish your thesis online,

%% * the second when you want to print your thesis to bind it, or
%% * the third when producing a ps file and a pdf/a from it.
%%
\documentclass[english, 12pt, a4paper, elec, utf8, a-1b, online]{aaltothesis}
%\documentclass[english, 12pt, a4paper, elec, utf8, a-1b]{aaltothesis}
%\documentclass[english, 12pt, a4paper, elec, dvips, online]{aaltothesis}

%% Use the following options in the \documentclass macro above:
%% your school: arts, biz, chem, elec, eng, sci
%% the character encoding scheme used by your editor: utf8, latin1
%% thesis language: english, finnish, swedish
%% make an archiveable PDF/A-1b or PDF/A-2b compliant file: a-1b, a-2b
%%                    (with pdflatex, a normal pdf containing metadata is
%%                     produced without the a-*b option)
%% typeset in symmetric layout and blue hypertext for online publication: online
%%            (no option is the default, resulting in a wide margin on the
%%             binding side of the page and black hypertext)
%% two-sided printing: twoside (default is one-sided printing)
%%

%% Use one of these if you write in Finnish (see the Finnish template
%% opinnaytepohja.tex)
%\documentclass[finnish, 12pt, a4paper, elec, utf8, a-1b, online]{aaltothesis}
%\documentclass[finnish, 12pt, a4paper, elec, utf8, a-1b]{aaltothesis}
%\documentclass[finnish, 12pt, a4paper, elec, dvips, online]{aaltothesis}

\usepackage{graphicx}
\usepackage{cite}

%% Math fonts, symbols, and formatting; these are usually needed
\usepackage{amsfonts,amssymb,amsbsy,amsmath}

%% Change the school field to specify your school if the automatically set name
%% is wrong
% \university{aalto-yliopisto}
% \school{Sähkötekniikan korkeakoulu}

%% Edit to conform to your degree programme
%%
\degreeprogram{Master's Programme in Automation and Electrical Engineering}
%%

%% Your major
%%
\major{Electronic and Digital Systems}
%%

%% Major subject code
%%
\code{ELEC3060}
%%
 
%% Choose one of the three below
%%
%\univdegree{BSc}
\univdegree{MSc}
%\univdegree{Lic}
%%

%% Your name (self explanatory...)
%%
\thesisauthor{Markus Säynevirta}
%%

%% Your thesis title comes here and possibly again together with the Finnish or
%% Swedish abstract. Do not hyphenate the title, and avoid writing too long a
%% title. Should LaTeX typeset a long title unsatisfactorily, you mght have to
%% force a linebreak using the \\ control characters.
%% In this case...
%% Remember, the title should not be hyphenated!
%% A possible "and" in the title should not be the last word in the line, it
%% begins the next line.
%% Specify the title again without the linebreak characters in the optional
%% argument in box brackets. This is done because the title is part of the 
%% metadata in the pdf/a file, and the metadata cannot contain linebreaks.
%%
\thesistitle{Uplink Interception of Very Small Aperture Satellite Terminals from Aerial Platforms}
%\thesistitle[Title of the thesis]{Title of\\ the thesis}
%%

%%
\place{Espoo}
%%

%% The date for the bachelor's thesis is the day it is presented
%%
\date{31.7.2023}
%%

%% Thesis supervisor
%% Note the "\" character in the title after the period and before the space
%% and the following character string.
%% This is because the period is not the end of a sentence after which a
%% slightly longer space follows, but what is desired is a regular interword
%% space.
%%
\supervisor{Asst. Prof.\ Jaan Praks}
%%

%% Advisor(s)---two at the most---of the thesis. Check with your supervisor how
%% many official advisors you can have.
%%
\advisor{M. Sc. (Tech) Tapio Savunen}
%\advisor{MSc Sarah Scientist}
%%

%% Aaltologo: syntax:
%% \uselogo{aaltoRed|aaltoBlue|aaltoYellow|aaltoGray|aaltoGrayScale}{?|!|''}
%% The logo language is set to be the same as the thesis language.
%%
\uselogo{aaltoBlue}{!}
%%

%% The English abstract:
%% All the details (name, title, etc.) on the abstract page appear as specified
%% above.
%% Thesis keywords:
%% Note! The keywords are separated using the \spc macro
%%
\keywords{For keywords choose\spc concepts that are\spc central to your\spc thesis}
%%

%% The abstract text. This text is included in the metadata of the pdf file as well
%% as the abstract page.
%%
\thesisabstract{
Your abstract in English. Keep the abstract short. The abstract explains your 
research topic, the methods you have used, and the results you obtained. In the 
PDF/A format of this thesis, in addition to the abstract page, the abstract text is 
written into the pdf file's metadata. Write here the text that goes into the 
metadata. The metadata cannot contain special characters, linebreak or paragraph 
break characters, so these must not be used here. If your abstract does not contain 
special characters and it does not require paragraphs, you may take advantage of 
the abstracttext macro (see the comment below). Otherwise, the metadata abstract 
text must be identical to the text on the abstract page.
}

%% Copyright text. Copyright of a work is with the creator/author of the work
%% regardless of whether the copyright mark is explicitly in the work or not.
%% You may, if you wish, publish your work under a Creative Commons license (see
%% creaticecommons.org), in which case the license text must be visible in the
%% work. Write here the copyright text you want. It is written into the metadata
%% of the pdf file as well.
%% Syntax:
%% \copyrigthtext{metadata text}{text visible on the page}
%% 
%% In the macro below, the text written in the metadata must have a \noexpand
%% macro before the \copyright special character, and macros (\copyright and
%% \year here) must be separated by the \ character (space chacter) from the
%% text that follows. The macros in the argument of the \copyrighttext macro
%% automatically insert the year and the author's name. (Note! \ThesisAuthor is
%% an internal macro of the aaltothesis.cls class file).
%% Of course, the same text could have simply been written as
%% \copyrighttext{Copyright \noexpand\copyright\ 2018 Eddie Engineer}
%% {Copyright \copyright{} 2018 Eddie Engineer}
%%
\copyrighttext{Copyright \noexpand\copyright\ \number\year\ \ThesisAuthor}
{Copyright \copyright{} \number\year{} \ThesisAuthor}

%% You can prevent LaTeX from writing into the xmpdata file (it contains all the 
%% metadata to be written into the pdf file) by setting the writexmpdata switch
%% to 'false'. This allows you to write the metadata in the correct format
%% directly into the file thesistemplate.xmpdata.
%\setboolean{writexmpdatafile}{false}

%% All that is printed on paper starts here
%%
\begin{document}

%% Create the coverpage
%%
\makecoverpage

%% Typeset the copyright text.
%% If you wish, you may leave out the copyright text from the human-readable
%% page of the pdf file. This may seem like a attractive idea for the printed
%% document especially if "Copyright (c) yyyy Eddie Engineer" is the only text
%% on the page. However, the recommendation is to print this copyright text.
%%
\makecopyrightpage

%% Note that when writting your thesis in English, place the English abstract
%% first followed by the possible Finnish or Swedish abstract.

%% Abstract text
%% All the details (name, title, etc.) on the abstract page appear as specified
%% above.
%%
\begin{abstractpage}[english]
  Your abstract in English. Keep the abstract short. The abstract explains your
  research topic, the methods you have used, and the results you obtained.  
  
  The abstract text of this thesis is written on the readable abstract page as
  well as into the pdf file's metadata via the $\backslash$thesisabstract macro
  (see above). Write here the text that goes onto the readable abstract page.
  You can have special characters, linebreaks, and paragraphs here. Otherwise,
  this abstract text must be identical to the metadata abstract text.
  
  If your abstract does not contain special characters and it does not require
  paragraphs, you may take advantage of the abstracttext macro (see the comment
  below).
\end{abstractpage}

%% The text in the \thesisabstract macro is stored in the macro \abstractext, so
%% you can use the text metadata abstract directly as follows:
%%
%\begin{abstractpage}[english]
%	\abstracttext{}
%\end{abstractpage}

%% Force a new page so that the possible Finnish or Swedish abstract does not
%% begin on the same page
%%
\newpage
%%
%% Abstract in Finnish.  Delete if you don't need it. 
%%
\thesistitle{LEO-megakonstellaatioiden salakuunteleminen}
\supervisor{Apul. prof.\ Jaan Praks}
\advisor{DI Tapio Savunen}
\degreeprogram{Elektroniikka ja sähkötekniikka}
%\department{Elektroniikan ja nanotekniikan laitos}
\major{Sopiva pääaine}
%% The keywords need not be separated by \spc now.
\keywords{Vastus, resistanssi, lämpötila}
%% Abstract text
\begin{abstractpage}[finnish]
  Tiivistelmässä on lyhyt selvitys
  kirjoituksen tärkeimmästä sisällöstä: mitä ja miten on tutkittu,
  sekä mitä tuloksia on saatu. 
\end{abstractpage}


%% Preface
%%
%% This section is optional. Remove it if you do not want a preface.
\mysection{Preface}
%\mysection{Esipuhe}
I want to thank Professor Pirjo Professori and my instructor Dr Alan Advisor for 
their good and poor guidance.\\

\vspace{5cm}
Espoo, 31.7.2023

\vspace{5mm}
{\hfill Markus\ Säynevirta \hspace{1cm}}

%% Force a new page after the preface
%%
\newpage


%% Table of contents. 
%%
\thesistableofcontents


%% Symbols and abbreviations
\mysection{Symbols and abbreviations}

\subsection*{Symbols}

\begin{tabular}{ll}
$\mathbf{B}$  & magnetic flux density  \\
$c$              & speed of light in vacuum $\approx 3\times10^8$ [m/s]\\
$\omega_{\mathrm{D}}$    & Debye frequency \\
$\omega_{\mathrm{latt}}$ & average phonon frequency of lattice \\
$\uparrow$       & electron spin direction up\\
$\downarrow$     & electron spin direction down
\end{tabular}

\subsection*{Operators}

\begin{tabular}{ll}
$\nabla \times \mathbf{A}$              & curl of vectorin $\mathbf{A}$\\
$\displaystyle\frac{\mbox{d}}{\mbox{d} t}$ & derivative with respect to 
variable $t$\\[3mm]
$\displaystyle\frac{\partial}{\partial t}$  & partial derivative with respect 
to variable $t$ \\[3mm]
$\sum_i $                       & sum over index $i$\\
$\mathbf{A} \cdot \mathbf{B}$    & dot product of vectors $\mathbf{A}$ and 
$\mathbf{B}$
\end{tabular}

\subsection*{Abbreviations}

\begin{tabular}{ll}
AC         & alternating current \\
APLAC      & an object-oriented analog circuit simulator and design tool \\
           & (originally Analysis Program for Linear Active Circuits) \\
BCS        & Bardeen-Cooper-Schrieffer \\ %% dash between the names
DC         & direct current \\
TEM        & transverse eletromagnetic
\end{tabular}


%% \clearpage is similar to \newpage, but it also flushes the floats (figures
%% and tables).
%%
\cleardoublepage

%% Text body begins. Note that since the text body is mostly in Finnish the
%% majority of comments are also in Finnish after this point. There is no point
%% in explaining Finnish-language specific thesis conventions in English.
%% This text will be translated to English soon.
%%
\section{Introduction (WIP)}
During the last decade, the satellite communications industry has entered into an era of change. The most prominent new trend is the large megaconstellations with hundreds to thousands of satellites in low earth orbit (LEO). These have been enabled by the falling costs in space launches and the mass-production of satellite hardware based on COTS technology.

Aside from commercial markets, governmental organisations, such as civilian public safety authorities and defence ministries, are looking into augmenting their existing connectivity infrastructure with commercial satcom services. Among a set of requirements, these organisations place a very stringent standard of security on the communications solutions they utilise.

Prior experimental research into the security of traditional geostationary broadband services has revealed serious security vulnerabilities. Known attack vectors include for example the eavesdropping of network traffic with widely available and relatively inexpensive television equipment. Rapidly growing number of users and limits in launch capacity are exposing bottlenecks in the throughput of LEO satcom networks. In the worst case scenario, this may tempt the new operators to follow the questionable practices of their predecessors in trading information security for gains in network performance.

Considering this prior history and the recent rapid growth, it is important to better understand the security aspects of this emerging technology. This thesis will start by delving into the methods of eavesdropping a satellite network. General security architecture of the new LEO broadband services will be discussed in relation to this attack vector. Possible vulnerabilities will be further explored via simulation and field experiments with the OneWeb satellite constellation. Overall, the topic will be discussed from the viewpoint of the public safety and defence user groups.

The thesis seeks to answer what kind of risk space-borne uplink eavesdropping poses to modern very small aperture terminal satellite communications. The eavesdroppers are assumed to be randomly distributed at a set of altitudes according to homogenous binomial point processes. Eavesdroppers are assumed to be passive in nature and to be not colluding with each other, i.e. the received signals are decoded individually.

Knowing the history of the field and the prior vulnerabilities with geostationary satcom networks, the current hypothesis is that there could be information leakage happening in the over-the-air communications of the constellation. It is important to understand whether this is happening and if so, to what extent, as it might be possible to extract sensitive user information from these transmissions.

The core goal of the thesis is to gain better understanding regarding the security of the over-the-air communications with modern LEO satcom constellations.

Areas of interest include the traffic flow security of the constellation and whether transmissions sent over it are possible to be set up in a way that avoids information leakage to adversarial groups. This is of paramount importance for the defence user groups, as the traffic patterns or information in the packet headers could reveal factors such as location, number or identity of an individual or a group of users.

%% Opinn\"aytteess\"a jokainen osa alkaa uudelta sivulta, joten \clearpage
%%
%% In a thesis, every section starts a new page, hence \clearpage
\clearpage

\section{Background}

\subsection{LEO megaconstellations}
\subsubsection{History and recent developments}
During the last five years the satellite communications industry has entered into an era of change. The most prominent new trend is the large megaconstellations with hundreds to thousands of satellites in low earth orbit (LEO). These systems have been enabled by the falling costs in space launches and the mass-production of satellite hardware based on COTS technology.

While unlikely to widely replace terrestrial solutions, satellite systems have the potential to serve as a complimentary coverage and capacity solution for both commercial and public safety users. These systems could play a part in the ongoing broadband transition of the existing critical communications networks. Public safety users have more stringent requirements for their communication services when compared to the best effort service provided to normal commercial users.

So far the furthest strides in the new telecom constellations have been made by four companies: SpaceX with its Starlink, OneWeb, Telesat and Amazon with its Project Kuiper. SpaceX and Amazon are U.S. companies and Telesat is Canadian, while OneWeb is controlled by its investors from India, the U.K., France and Japan. In addition to them, multiple other actors from around the world have expressed interest in similar projects. These include for example the EU’s Secure Connectivity Initiative and the Chinese Guo Wang constellation.

All four projects furthest in development have significant funding behind their concepts and have secured the necessary regulatory approvals for the initial deployments of their systems.

Multiple LEO megaconstellations are currently in the design and deployment phase, of which Starlink, OneWeb, Telesat and Kuiper are farthest in the development and deployment. These constellations are operating on dedicated bands and are likely the most viable near term solution. They are based on vendor-specific vendor-specific user terminals working as WiFi routers that relay the communications on Ka and Ku-band frequencies to the satellite constellation.

Two US companies, Lynk and AST SpaceMobile, are also planning on beaming broadband service from orbit directly to smartphone sized handsets on 5G frequencies. The latter services are less demonstrated and will need significant R\&D investment before becoming a viable option, while the prior are already reaching commercial operability in limited geographic regions.

\subsubsection{Key technical characteristics}
The LEO altitude leads to significantly lower latency and the large number of satellites allows for relatively high overall data throughput when compared with the earlier satellite systems but still significantly lower when compared to terrestrial systems. While NGSO constellations are nothing new, the emerging operators are promising to offer magnitudes better broadband service when compared to the earlier services offered by e.g. SES O3b and Iridium while providing the services also at a price point that is competitive with other forms of connectivity [1].

The services are built around vendor-specific user terminals working as WiFi routers that relay the communications on dedicated Ka and Ku-band frequencies to the satellite constellation.

\subsubsection{Example system architechtures (OneWeb / Starlink)}
%%%%%%%%%%%%%%%%%%  APPENDIX: COMPARISON TABLE  %%%%%%%%%%%%%%%%%%
%%%%%%%%%%%%%%%%%%  FIGURE: ONEWEB NETWORK ARCHITECTURE HERE  %%%%%%%%%%%%%%%%%%
The space segment of the OneWeb system comprises a megaconstellation of 648 LEO satellites distributed into 12 polar orbital planes of 49 evenly spaced satellites, as well as a number of in-orbit spares. Operational satellites fly in an inclined polar orbit with an altitude of 1200 km. Each satellite transmits and receives user terminal (UT) traffic via its 16 fixed Ku-band beams, each of which covers a geographic area with dimensions of 1600 km in longitude and 65 km in latitude. Gateway traffic is forwarded to the satellite network portals (SNP) via two identical steerable Ka-band spot beams with a significantly more focused circular coverage pattern. \cite{henri2020oneweb, worldvu2016loi}

Earth Stations of the OneWeb system can be broadly divided into three categories: tracking, telemetry and control (TT\&C) sites,  gateways and user terminals (UTs). In the following, we will focus on the two latter ones, as they are integral to describing the end-to-end configuration of the OneWeb network. \cite{worldvu2016loi}

Going deeper into the gateway-side architecture, the infrastructure can be further split into three components, which are network data centres (NDC), points-of-presence (PoP) and satellite network portals (SNP). NDCs host the authentication, authorization, policy and UT databases and are deployed in key global locations. PoPs connect the OneWeb network to the Internet and are deployed at key Internet peering points. Finally, SNPs maintain the connectivity to the LEO space segment composed of the OneWeb satellite constellation. They are situated in remote locations around the globe with room for large antenna arrays of 7 to 30 full motion antennas (on average 16) equipped with a 3.5 m Ka-band dish. \cite{henri2020oneweb}

On the user terminal side, a similar architectural breakdown can be made – the terminal consists of a satellite antenna, receiver and a customer network exchange (CNX) router. The latter connects the terminal to the end-user devices such as laptops or smartphones. 
\cite{henri2020oneweb} RF transmissions received by the satellite antenna are demodulated and converted to a digital data stream by the receiver hardware of the terminal.

As OneWeb is a LEO satellite system, UTs need to track the movements of the orbiting satellites in real-time and handover between them as they move in and out of view in order to maintain constant connectivity. This can be achieved either with traditional steerable dish or more modern phased array antenna designs. With the prior, two apertures may need to be employed for uninterrupted connectivity, as retrace speed of a single aperture is the inherent limiting factor for hand-over time between satellites. On the other hand, phased array antennas require only a single aperture as their electronic switching can be considered almost instantaneous. \cite{worldvu2016loi}

Continuing with the distinguishing qualities of the OneWeb system, maybe the most significant is the nature of its air interface coverage pattern, also known as the cell layout. In the OneWeb satellite RAN, the cells are inherently varying and mobile, while on the contrary they are practically geographically static and pre-defined in a terrestrial network of fixed eNBs. Consequently, the movement of the UTs (for example equipment mounted on an aircraft or a high-speed train) is relatively slow when compared to the relative velocities of the satellites in orbit. This means that UT handovers happen mostly due to the orbital movement of the satellites rather than the movement of the UT relative to the surface of the earth, which is the dominating cause of UE handovers in terrestrial systems. \cite{corson2019admission}

In addition to their moving nature, satellite cells are significantly larger in their coverage area when compared to their terrestrial counterparts. This has multiple consequences for 
 \cite{corson2019admission}

 OneWeb satellite system makes use of a bent pipe architecture for both its forward and return links. In the forward direction, each Ku-band user terminal downlink maps onto a predetermined Ka-band gateway uplink and vice-versa in the return direction. \cite{worldvu2016loi, portillo2019technical}

OneWeb satellite system makes use of a bent pipe architecture for both its forward and return links. In the forward direction, each Ku-band user terminal downlink maps onto a predetermined Ka-band gateway uplink and vice-versa in the return direction. \cite{worldvu2016loi, portillo2019technical}

\subsection{Aerial Platforms}
\subsubsection{Technical capabilities}
\subsubsection{Key trade-offs}

\subsection{Communications intelligence}
Communications intelligence (COMINT) is often used as a synonym of signals intelligence (SIGINT) but it is actually a subfield of that broader area, which also includes electronics intelligence (ELINT). Both of them share a common set of methods. 

\subsubsection{Signal detection}
\subsubsection{Direction finding and radiopositioning}
\subsubsection{Eavesdropping and traffic analysis}
Fundamentally, secure communications rely on two core objectives being fulfilled. The intended receiver should be able to recover the original message without errors, while nobody else should be able to acquire any of the contained information. As is customary in cryptography, the transmitter is often referred to as Alice, the receiver as Bob and the eavesdropper as Eve. \cite{bloch2011physical}

This core principle of secure communications was formalised by Shannon \cite{shannon1949communication} in his 1949 paper through the notion of perfect secrecy achieved through a one-time pad. Shannon’s secrecy system assumes that both the intended recipient and the eavesdropper acquire the encoded codeword without any degradation, i.e. the communication channel is error-free. This theoretical assumption applies very rarely to real world systems, where some noise is almost always present. \cite{bloch2011physical}

Wyner \cite{wyner1975thewiretap} expanded on Shannon’s original system by exploring the role of noise in the context of secure communications through the channel model called \textit{degraded wiretap channel} (DWTC). The model assumes a situation where the sender (Alice) attempts to communicate with the legitimate recipient (Bob) over a noisy channel. Simultaneously an eavesdropper (Eve) observes a degraded version of the signal received by the legitimate recipient. \cite{barros2006secrecy}

Wyner’s wiretap channel introduced many mathematical tools for modelling information-theoretic security without the added complexity of fully general channel models. One of these important concepts is the secrecy capacity of the channel, which describes the greatest amount of information that can be confidentially communicated between the legitimate transmitter and receiver from the information-theoretic secrecy perspective. \cite{bloch2011physical}

Csiszár and Körner \cite{csiszar1978broadcast} developed a more general model that they termed the \textit{broadcast model with confidential messages}. 

\clearpage

\subsection{Threat model}
\subsubsection{Passive and active eavesdropping}
Space uplink
Ground downlink

The eavesdropper can act both passively and actively. In the prior case 

Regarding these two vectors, the research community has been thus far more focused on securing downlink communications from satellites to user terminals and gateways. Here, eavesdroppers have been assumed to be ground based, as space or airborne RF monitoring equipment has been seen as relatively limited in its performance compared to the terrestrial counterparts.

\subsubsection{Jamming}

\subsubsection{Active eavesdropping}

\subsubsection{Signal geolocation}

\subsection{Link budgets}
\subsection{Channel models}

\clearpage

\section{Research material and methods}

\subsection{Analysis toolchain}
\subsubsection{Aerospace Toolbox}
\subsubsection{Modelling of the HAPS}
\subsubsection{Modelling of the satellite constellation}
\subsubsection{Channel model}

\section{Threat scenarios in relation}
T\"ass\"a osassa kuvataan k\"aytetty tutkimusaineisto ja
tutkimuksen metodologiset valinnat, sek\"a
kerrotaan tutkimuksen toteutustapa ja k\"aytetyt menetelm\"at.


\clearpage

\section{Results}
\subsubsection{Passive eavesdropping}
\subsubsection{Active eavesdropping}
\subsubsection{Jamming}
\subsubsection{Radiolocation}


T\"ass\"a osassa esitet\"a\"an tulokset ja vastataan tutkielman alussa
esitettyihin tutkimuskysymyksiin. Tieteellisen kirjoitelman
arvo mitataan t\"ass\"a osassa esitettyjen tulosten perusteella. 

%% Huomaa seuraavassa kappaleessa lainausmerkkien ulkopuolella piste, 
%% koska piste ei lopeta lainattua tekstinp\"atk\"a\"a.
%% Jos lainattu tekstinp\"atk\"a loppuu v\"alimerkkiin, tulee v\"alimerkki
%% lainausmerkkien sis\"alle: 
%% "Et tu, Brute?" sanoi Caesar kuollessaan.
Tutkimustuloksien merkityst\"a on aina syyt\"a arvioida ja tarkastella
kriittisesti.  Joskus tarkastelu voi olla t\"ass\"a osassa, mutta se
voidaan my\"os j\"att\"a\"a viimeiseen osaan, jolloin viimeisen osan nimeksi
tulee >>Tarkastelu>>. Tutkimustulosten merkityst\"a voi arvioida my\"os
>>Johtop\"a\"at\"okset>>-otsikon alla viimeisess\"a osassa. 

T\"ass\"a osassa on syyt\"a my\"os arvioida tutkimustulosten luotettavuutta.
Jos tutkimustulosten merkityst\"a arvioidaan >>Tarkastelu>>-osassa,
voi luotettavuuden arviointi olla my\"os siell\"a. 

\section{Discussion}

\clearpage

\section{Conclusion}

\clearpage
%% L\"ahdeluettelo

\thesisbibliography

\bibliographystyle{ieeetran}
\bibliography{main}

%% Appendices
%% If you don't have appendices, remove \clearpage and \thesisappendix below.
\clearpage

\thesisappendix

\section{Esimerkki liitteest\"a\label{LiiteA}}

Liitteet eiv\"at ole opinn\"aytteen kannalta v\"altt\"am\"att\"omi\"a ja 
opinn\"aytteen tekij\"an on 
kirjoittamaan ryhtyess\"a\"an hyv\"a ajatella p\"arj\"a\"av\"ans\"a ilman liitteit\"a.
Kokemattomat kirjoittajat, jotka ovat huolissaan
tekstiosan pituudesta, paisuttavat turhan 
helposti liitteit\"a pit\"a\"akseen tekstiosan pituuden annetuissa rajoissa.
T\"all\"a tavalla ei synny hyv\"a\"a opinn\"aytett\"a.   

Liite on itsen\"ainen kokonaisuus, vaikka se t\"aydent\"a\"akin tekstiosaa.
Liite ei siten ole pelkk\"a listaus, kuva tai taulukko, vaan 
liitteess\"a selitet\"a\"an aina sis\"all\"on laatu ja tarkoitus. 

Liitteeseen voi laittaa esimerkiksi listauksia. Alla on 
listausesimerkki t\"am\"an liitteen luomisesta. 

%% Verbatim-ymp\"arist\"o ei muotoile tai tavuta teksti\"a. Fontti on monospace.
%% Verbatim-ymp\"arist\"on sis\"all\"a annettuja komentoja ei LaTeX k\"asittele. 
%% Vasta \end{verbatim}-komennon j\"alkeen jatketaan k\"asittely\"a.
\begin{verbatim}
	\clearpage
	\appendix
	\addcontentsline{toc}{section}{Liite A}
	\section*{Liite A}
	...
	\thispagestyle{empty}
	...
	teksti\"a
	...
	\clearpage
\end{verbatim}

Kaavojen numerointi muodostaa liitteiss\"a oman kokonaisuutensa:
\begin{align}
d \wedge A &= F, \label{liitekaava1}\\
d \wedge F &= 0. \label{liitekaava2}
\end{align}


\clearpage
\section{Toinen esimerkki liitteest\"a\label{LiiteB}}

%% Liitteiden kaavat, taulukot ja kuvat numeroidaan omana kokonaisuutenaan
%%
%% Equations, tables and figures have their own numbering in Appendices
%\renewcommand{\theequation}{B\arabic{equation}}
%\setcounter{equation}{0}  
%\renewcommand{\thefigure}{B\arabic{figure}}
%\setcounter{figure}{0}
%\renewcommand{\thetable}{B\arabic{table}}
%\setcounter{table}{0}

Liitteiss\"a voi my\"os olla kuvia, jotka
eiv\"at sovi leip\"atekstin joukkoon:
%% Ymp\"arist\"on figure parametrit htb pakottavat
%% kuvan t\"ah\"an, eik\"a LaTeX yrit\"a siirrell\"a niit\"a
%% hyv\"aksi katsomaansa paikkaan. 
%% Ymp\"arist\"o\"a center voi k\"aytt\"a\"a \centering-
%% komennon sijaan
%%
%% Example of a figure, note the use of htb parameters which force
%% the figure to be inserted here
%%
Liitteiden taulukoiden numerointi on kuvien ja kaavojen kaltainen:
\begin{table}[htb]
\caption{Taulukon kuvateksti.}
\label{liitetaulukko}
\begin{center}
\fbox{
\begin{tabular}{lp{0.5\linewidth}}
9.00--9.55  & K\"aytett\"avyystestauksen tiedotustilaisuus (osanottajat
ovat saaneet s\"ahk\"opostitse valmistautumisteht\"av\"at, joten tiedotustilaisuus
voidaan pit\"a\"a lyhyen\"a).\\
9.55--10.00 & Testausalueelle siirtyminen
\end{tabular}}
\end{center}
\end{table}
Kaavojen numerointi muodostaa liitteiss\"a oman kokonaisuutensa:
\begin{align}
T_{ik} &= -p g_{ik} + w u_i u_k + \tau_{ik},  \label{liitekaava3} \\
n_i    &= n u_i + v_i.                      \label{liitekaava4}
\end{align}

\end{document}
